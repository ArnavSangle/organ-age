\documentclass[10pt,twocolumn]{article}

% ============================================================
% Page layout (two-column)
% ============================================================
\usepackage[
  left=0.75in,
  right=0.75in,
  top=1in,
  bottom=1in,
  columnsep=0.25in
]{geometry}

% ============================================================
% Colors
% ============================================================
\usepackage[dvipsnames]{xcolor}

% ---- Stroke / accent colors ----
\definecolor{OAslate}{HTML}{334155} % deep slate (strokes/text)
\definecolor{OAamber}{HTML}{B45309} % warm amber
\definecolor{OAteal}{HTML}{0F766E}  % teal

% Lighter blue/rose
\definecolor{OAblue}{HTML}{3B82F6}
\definecolor{OArose}{HTML}{F43F5E}

% Stronger blue for links
\definecolor{OAblueLink}{HTML}{2563EB}

% ---- Fill colors (very light) ----
\definecolor{OAfillRose}{HTML}{FCE7F3}
\definecolor{OAfillAmber}{HTML}{FFEDD5}
\definecolor{OAfillSage}{HTML}{ECFDF3}
\definecolor{OAfillTeal}{HTML}{E6FFFB}
\definecolor{OAfillBlue}{HTML}{EFF6FF}
\definecolor{OAfillSlate}{HTML}{F1F5F9}

% ---- Backward-compatible aliases ----
\colorlet{OAgray}{OAfillSlate}
\colorlet{OAcream}{OAfillAmber}
\colorlet{OAmint}{OAfillSage}
\colorlet{OAblueFill}{OAfillBlue}
\colorlet{OAroseFill}{OAfillRose}

% ============================================================
% Typography & spacing
% ============================================================
\usepackage{lmodern}
\usepackage[T1]{fontenc}
\usepackage[utf8]{inputenc}
\usepackage[english]{babel}
\usepackage{microtype}
\microtypesetup{protrusion=true, expansion=true}

\tolerance=1000
\emergencystretch=2em
\hyphenpenalty=500
\exhyphenpenalty=500

\setlength{\parindent}{1em}
\setlength{\parskip}{0.3em}

\usepackage{titling}
\setlength{\droptitle}{-3em}

% ============================================================
% Section spacing
% ============================================================
\usepackage{titlesec}
\titlespacing*{\section}{0pt}{0.8em}{0.4em}
\titlespacing*{\subsection}{0pt}{0.6em}{0.25em}
\titlespacing*{\subsubsection}{0pt}{0.4em}{0.2em}

\usepackage{etoolbox}
\makeatletter
\patchcmd{\@startsection}
  {\@afterindenttrue}
  {\@afterindentfalse}
  {}
  {}
\makeatother

% ============================================================
% Math / tables
% ============================================================
\usepackage{amsmath, amssymb}
\usepackage{booktabs}

% ============================================================
% Figures / floats (two-column)
% ============================================================
\usepackage{graphicx}
\usepackage{caption}
\usepackage{subcaption}
\usepackage{dblfloatfix}
\usepackage[section]{placeins}
\usepackage{float}

\setkeys{Gin}{width=\columnwidth, keepaspectratio}

\graphicspath{
  {figures/paper/}
}

\setcounter{topnumber}{2}
\setcounter{dbltopnumber}{2}
\renewcommand{\topfraction}{0.9}
\renewcommand{\dbltopfraction}{0.9}
\renewcommand{\textfraction}{0.08}
\renewcommand{\floatpagefraction}{0.85}
\renewcommand{\dblfloatpagefraction}{0.85}

\captionsetup{
  font=small,
  labelfont=bf,
  textfont=normalfont
}

% ============================================================
% TikZ
% ============================================================
\usepackage{tikz}
\usetikzlibrary{arrows.meta, positioning, fit}

% IMPORTANT:
% - Do NOT define a style named "step" (TikZ has a key "step=<len>").
% - Use OAstep / OAin instead.
\tikzset{%
  OAarrow/.style={-{Latex[length=2.2mm]}, line width=0.55pt, draw=OAslate},%
  OAin/.style={%
    draw=OAslate!70, rounded corners=2.2mm, line width=0.40pt, align=center,
    inner xsep=6pt, inner ysep=4.5pt, minimum width=0.27\columnwidth,
    fill=OAfillSlate
  },%
  OAstep/.style={%
    draw=OAslate, rounded corners=2.2mm, line width=0.45pt, align=center,
    inner xsep=6.5pt, inner ysep=5.5pt, minimum width=0.86\columnwidth
  },%
  stageRose/.style={fill=OAfillRose},%
  stageAmber/.style={fill=OAfillAmber},%
  stageSage/.style={fill=OAfillSage},%
  stageTeal/.style={fill=OAfillTeal},%
  stageBlue/.style={fill=OAfillBlue},%
  stageSlate/.style={fill=OAfillSlate}%
}

% ============================================================
% Citations + hyperlinks (hyperref late)
% ============================================================
\usepackage{cite}
\usepackage{hyperref}
\hypersetup{
  colorlinks=true,
  linkcolor=OAblueLink,
  citecolor=OAblueLink,
  urlcolor=OAblueLink
}

% ============================================================
% Title and Author
% ============================================================
\title{Organ-Age: Multimodal Fusion of Transcriptomic and Radiological Signals for Organ-Resolved Biological Age Estimation}
\author{
  Arnav Sangle, Independent Researcher, Ranchview High School, Irving TX, USA \\
  \texttt{arnavsangle08@gmail.com}
}
\date{}

\begin{document}

% Full-width title + abstract block in a two-column paper
\makeatletter
\twocolumn[
\begin{@twocolumnfalse}
\maketitle
\vspace{-2em}
\noindent\rule{\textwidth}{0.4pt}
\vspace{0em}

\begin{abstract}
Biological aging does not proceed uniformly across the body. Different organs accumulate
molecular, structural, and physiological damage at different rates, yet most biological age
predictors rely on a single data type and consequently collapse aging into a single,
organism-wide estimate. This compression limits clinical interpretability and obscures organ-level
trajectories that span both gene regulation and tissue morphology.

Here I present Organ-Age, a framework that estimates organ-resolved biological age by fusing
transcriptomic and radiological signals. Using bulk RNA-seq from GTEx, chest radiographs from
CheXpert, and structural brain MRI from IXI, the model first encodes each modality into a
fixed-length embedding, aligns embeddings into a shared latent space via an InfoNCE-style
contrastive objective, and then merges them through a transformer-based fusion module to
produce organ-level age predictions with uncertainty. On a combined cohort of over 190{,}000
samples, the aligned multimodal model achieves a mean absolute error of $\sim$9.3 years and
tracks chronological age closely across the adult lifespan.

Residual deviations between predicted organ age and chronological age---organ-age deltas---expose
structured acceleration and deceleration patterns that differ by organ and are consistent with
known tissue-level biology. These findings suggest that biological age is better modeled as a
multi-signal, organ-resolved construct than as a single number and that alignment plus attention-based
fusion is a viable route toward interpretable organ-specific aging assessment.
\end{abstract}

\vspace{1em}
\noindent\rule{\textwidth}{0.4pt}
\vspace{1em}
\end{@twocolumnfalse}
]
\makeatother

% ============================================================
% INTRODUCTION
% ============================================================
\section{Introduction}

Biological aging involves progressive functional decline at the molecular, cellular, and organ
levels. Chronological age is a crude proxy for this process: individuals of the same calendar
age vary widely in healthspan, disease burden, and rate of physiological deterioration.
Recognizing this, a growing body of work has sought to estimate biological age, a quantity
that reflects physiological state rather than time since birth \cite{horvath2013,hannum2013}.

The most established biological age predictors are epigenetic clocks, which use DNA methylation
patterns to estimate a single, organism-wide age value. These clocks correlate with mortality,
chronic disease risk, and functional decline \cite{levine2018}. They have been instrumental in
showing that aging is, in principle, quantifiable from molecular data. However, because they
collapse aging into one number, they say little about how individual tissues or organs age
relative to one another---a distinction that matters clinically, since age-related pathology
is often organ-specific.

A growing literature confirms that aging is not synchronous across the body. Lungs in chronic
smokers and livers in heavy drinkers can deteriorate well ahead of other organs in the same
individual. More broadly, tissue-resolved profiling studies document wide variation in the pace
and character of age-associated molecular change across organ systems \cite{zakarpolyak2023}.
Despite this, few computational approaches attempt to estimate organ-specific biological age
directly. Most existing models work from a single data type---gene expression, imaging, or
methylation---and therefore capture only part of the picture.

On the imaging side, deep learning models trained on chest radiographs and brain MRI show that
macrostructural aging features---changes in skeletal morphology, soft-tissue composition, and
brain volume---can be inferred from medical images with reasonable accuracy
\cite{cole2017,rajpurkar2018chexnet}. These structural cues are largely invisible to molecular
clocks, and vice versa. Yet imaging-based age predictors are typically developed in isolation,
with no explicit connection to underlying molecular state.

Multimodal learning offers a natural way to bridge this gap. Contrastive objectives such as
InfoNCE can pull embeddings from different data types into a common representation space,
preserving structure within each modality while encouraging cross-modal agreement
\cite{oord2018cpc,chen2020simclr,radford2021clip}. Transformer-based architectures add the
ability to model interactions across modalities through attention \cite{vaswani2017}, and recent
systems like Perceiver IO and FLAVA demonstrate that diverse inputs can be merged through a
shared latent bottleneck without requiring pixel-level or token-level alignment
\cite{jaegle2021perceiver,singh2021flava}. Separately, probabilistic regression emphasizes the
value of predicting uncertainty alongside point estimates, which is relevant for targets like
biological age that are inherently noisy and never directly observed \cite{kendall2017}. Taken
together, these lines of work suggest that biological age may be better understood as a composite
quantity shaped by the interplay of molecular and structural processes than as something derivable
from any single measurement.

With this motivation, I present Organ-Age, a framework for estimating organ-level biological age
by jointly modeling transcriptomic and radiological signals. Separate encoders produce embeddings
from RNA-seq data, chest radiographs, and structural MRI scans; these embeddings are pulled into
a shared latent space through contrastive alignment and then merged by a transformer-based fusion
module to yield organ-specific age predictions. A central output of the work is the residual
deviation between predicted organ age and chronological age, which can flag accelerated or
decelerated aging in specific tissues.

\begin{figure}[t]
\centering
\begin{tikzpicture}[
  font=\small,
  node distance=4.6mm
]

% --- Top inputs (compact) ---
\node[OAin] (rna) {RNA\\(GTEx)};
\node[OAin, right=4.5mm of rna] (xray) {X-ray\\(CheXpert)};
\node[OAin, right=4.5mm of xray] (mri) {MRI\\(IXI)};

% --- Vertical pipeline ---
\node[OAstep, stageAmber, below=7.5mm of xray] (enc)
  {\textbf{Modality Encoders}\\{\footnotesize (learn embeddings)}};

\node[OAstep, stageTeal, below=of enc] (align)
  {\textbf{Contrastive Alignment}\\{\footnotesize (shared latent space)}};

\node[OAstep, stageBlue, below=of align] (fuse)
  {\textbf{Fusion Transformer}\\{\footnotesize (cross-modal attention)}};

\node[OAstep, stageRose, below=of fuse] (pred)
  {\textbf{Organ-Age Prediction}\\{\footnotesize $\hat{y}$ and uncertainty}};

\node[OAstep, stageSlate, below=of pred] (gap)
  {\textbf{Organ-age gap}\\{\footnotesize $\Delta=\hat{y}-y$}};

% --- Arrows (inputs into first stage) ---
\draw[OAarrow] (rna.south) -- ++(0,-3.2mm) -| (enc.north);
\draw[OAarrow] (xray.south) -- (enc.north);
\draw[OAarrow] (mri.south) -- ++(0,-3.2mm) -| (enc.north);

% --- Vertical arrows ---
\draw[OAarrow] (enc) -- (align);
\draw[OAarrow] (align) -- (fuse);
\draw[OAarrow] (fuse) -- (pred);
\draw[OAarrow] (pred) -- (gap);

\end{tikzpicture}
\caption{Conceptual overview of Organ-Age: modality-specific encoders, alignment, fusion,
and organ-age prediction with uncertainty and age-gap outputs.}
\label{fig:conceptual_overview}
\end{figure}

Figure~\ref{fig:conceptual_overview} outlines the Organ-Age pipeline. The core idea is that
organ-specific biological age reflects the joint contribution of molecular and structural
signals and that aligning representations from these domains can surface aging patterns
invisible to any single modality. Beyond population-level evaluation, Organ-Age also supports
per-subject interpretation through individualized organ-age panels (v4) and gene-level
attribution summaries (v4.5), which trace organ-age deviations back to candidate transcriptomic
drivers.

% ============================================================
% METHODS
% ============================================================
\section{Methods}

Three publicly available datasets supply the molecular and structural aging signals used in
this work. Transcriptomic data come from the Genotype-Tissue Expression (GTEx) v10 release,
which contains bulk RNA-seq profiles across dozens of human tissues sampled from adult donors
\cite{gtex2017}. Radiological data come from CheXpert, a large collection of frontal chest
radiographs with associated demographic metadata \cite{irvin2019chexpert}. Structural brain
imaging comes from the IXI dataset, which provides T1-weighted MRI scans from healthy adults
spanning a broad age range \cite{ixi2020}. Samples without reliable age labels or that failed
quality-control filters were dropped before any modeling.

Each data type was preprocessed according to standard domain-specific protocols. Gene
expression counts were variance-stabilized following the DESeq2 approach \cite{deseq2}, and
batch effects from site and technical sources were corrected with ComBat \cite{combat2007}.
Chest radiographs were resized and intensity-normalized, then passed through convolutional
backbones initialized from pretrained ResNet weights \cite{he2016resnet}; transfer-learning
practices from BiT \cite{kolesnikov2020bit} guided the fine-tuning strategy. MRI volumes
were skull-stripped, intensity-normalized, and nonlinearly registered to a common template
with ANTs \cite{avants2007ants}, supplemented by standard FSL preprocessing steps
\cite{smith2004fsl}.

A separate encoder network handles each modality. For transcriptomics, a feedforward network
maps normalized expression vectors to a fixed-length embedding. For chest X-rays and MRI
scans, convolutional networks---adapted to 2-D and 3-D inputs, respectively---serve the
same purpose. All three encoders output embeddings of the same dimensionality so that they
can be compared and merged downstream.

To bring modality-specific embeddings into a common space, I appended learned projection heads
to each encoder and trained them with an InfoNCE-style contrastive objective \cite{oord2018cpc},
borrowing design choices from SimCLR and CLIP \cite{chen2020simclr,radford2021clip}. Pairwise
similarity is measured by cosine similarity, which depends on the angle between vectors and is
insensitive to their magnitude:
\begin{equation}
\mathrm{sim}(u, v) = \frac{u^\top v}{\|u\| \|v\|}
\label{eq:cosine}
\end{equation}

Alignment was optimized by minimizing the contrastive loss
\begin{equation}
\mathcal{L}_{\text{InfoNCE}}
=
- \log
\frac{\exp(\mathrm{sim}(u,u') / \tau)}
{\sum_{j} \exp(\mathrm{sim}(u,u_j)/\tau)}
\label{eq:infonce}
\end{equation}
which pulls embeddings from the same biological context closer together while pushing apart
those from unrelated samples. The purpose is not to erase modality-specific information but to
reduce modality identity as the dominant axis of variation, so that age-related geometry can be
fused reliably.

\begin{figure}[t!]
\centering
\begin{tikzpicture}[
  font=\small,
  inmod/.style={
    draw=OAslate!70, rounded corners=3pt, align=center, inner sep=4pt,
    minimum width=2.1cm, minimum height=0.85cm, fill=OAfillSlate
  },
  enc/.style={
    draw=OAslate, rounded corners=4pt, align=center, inner sep=6pt,
    minimum width=2.9cm, minimum height=0.95cm, fill=OAfillAmber
  },
  proj/.style={
    draw=OAslate, rounded corners=4pt, align=center, inner sep=6pt,
    minimum width=2.9cm, minimum height=0.95cm, fill=OAfillTeal
  },
  loss/.style={
    draw=OAslate, rounded corners=4pt, align=center, inner sep=6pt,
    minimum width=2.9cm, minimum height=0.95cm, fill=OAfillRose
  },
  arrow/.style={-Stealth, thick, draw=OAslate!80},
  dashedarrow/.style={-Stealth, thick, draw=OAslate!60, dashed}
]

\node[inmod] (rna) {RNA\\$x_{\mathrm{RNA}}$};
\node[inmod, right=0.65cm of rna] (xray) {X-ray\\$x_{\mathrm{XR}}$};
\node[inmod, right=0.65cm of xray] (mri) {MRI\\$x_{\mathrm{MRI}}$};

\node[enc,  below=0.55cm of xray] (encs)  {Encoders\\$f_m(\cdot)$};
\node[proj, below=0.50cm of encs] (projh) {Projection heads\\$g_m(\cdot)$};
\node[loss, below=0.50cm of projh] (cl)   {Contrastive loss\\{\footnotesize (shared latent)}};

\draw[arrow] (rna)  -- (encs);
\draw[arrow] (xray) -- (encs);
\draw[arrow] (mri)  -- (encs);
\draw[arrow] (encs) -- (projh);
\draw[arrow] (projh) -- (cl);

% Vertical note box under RNA (rotate BOX ONLY; text stays upright)
\node[
  draw=OAslate!55,
  rounded corners=3pt,
  fill=OAfillSlate,
  align=left,
  inner sep=5pt,
  text width=1.55cm,
  minimum height=3.05cm,
  shape border rotate=90,
  below=0.95cm of rna
] (note)
{\textbf{Goal:} modality-invariant\\age structure \,+ robust fusion};

\draw[dashedarrow]
  (note.south) -- ++(0,-0.25) |- (cl.west);

\end{tikzpicture}

\caption{
\textbf{Contrastive representation alignment.}
Each modality is encoded and passed through a projection head; a contrastive objective
encourages embeddings from different modalities to occupy a shared latent space that preserves
age-related structure while reducing modality-driven separation.
}
\label{fig:contrastive_alignment}
\end{figure}

Figure~\ref{fig:contrastive_alignment} diagrams this alignment step.

Once aligned, modality embeddings are combined by a transformer-based fusion module. Instead of
simple concatenation or prediction averaging, the fusion model uses attention to learn cross-modal
interactions, letting molecular and radiological information contribute in a data-dependent way.
Given aligned embeddings $\{z^{(1)}, z^{(2)}, \dots, z^{(M)}\}$ from $M$ modalities, the fused
representation is
\begin{equation}
z_{\text{fusion}} = f_{\text{Transformer}}\left(
z^{(1)}, z^{(2)}, \dots, z^{(M)}
\right)
\label{eq:fusion}
\end{equation}
The design follows Perceiver IO and FLAVA \cite{jaegle2021perceiver,singh2021flava}, adapted to
operate on continuous biomedical embeddings.

\begin{figure}[t!]
\centering
\begin{tikzpicture}[
  font=\small,
  node distance=4.8mm
]

\node[OAstep, stageBlue] (pred)
  {\textbf{Predict}\\{\footnotesize $\hat{y}$ \,+ uncertainty}};

\node[OAstep, stageAmber, below=of pred] (cal)
  {\textbf{Calibrate}\\{\footnotesize $\hat{y}\rightarrow \hat{y}_{\mathrm{cal}}$}};

\node[OAstep, stageTeal, below=of cal] (ci)
  {\textbf{CI / intervals}\\{\footnotesize (interpretability)}};

\node[OAstep, stageSage, below=of ci] (gap)
  {\textbf{Age-gap}\\{\footnotesize $\Delta=\hat{y}_{\mathrm{cal}}-y$}};

\node[OAstep, stageRose, below=of gap] (z)
  {\textbf{Standardize}\\{\footnotesize $z=\Delta/\sigma_{\mathrm{organ}}$}};

\node[OAstep, stageSlate, below=of z] (panel)
  {\textbf{Subject panel}\\{\footnotesize (organ-level summary)}};

\draw[OAarrow] (pred) -- (cal);
\draw[OAarrow] (cal) -- (ci);
\draw[OAarrow] (ci) -- (gap);
\draw[OAarrow] (gap) -- (z);
\draw[OAarrow] (z) -- (panel);

\end{tikzpicture}
\caption{Post-processing and interpretability pathway: calibration, uncertainty intervals,
age-gap computation, standardization, and subject-level organ summaries.}
\label{fig:fusion_transformer}
\end{figure}

Figure~\ref{fig:fusion_transformer} shows the post-fusion processing pipeline.

The fused representation is fed to a probabilistic regression head that outputs both a point
estimate and a learned uncertainty term, rather than a bare scalar prediction. Concretely, the
head produces
\begin{equation}
(\mu, \sigma^2) = g_{\theta}(z_{\text{fusion}})
\label{eq:age_prediction}
\end{equation}
where $g_{\theta}(z_{\text{fusion}})$ is the regression head with parameters $\theta$. The
training objective is a Gaussian negative log-likelihood \cite{kendall2017}
\begin{equation}
\mathcal{L}_{\text{NLL}}
=
\frac{1}{2}\log\sigma^2
+
\frac{(\mu-y)^2}{2\sigma^2}
\label{eq:gnll}
\end{equation}
so that the model is incentivized to widen its uncertainty estimate when its predictions are
poor, rather than absorbing all error into the mean.

Training follows a staged curriculum: first the modality-specific encoders are trained
independently, then the contrastive alignment and fusion stages are optimized jointly. The
combined loss is
\begin{equation}
\mathcal{L}_{\text{total}}
=
\lambda_{\text{align}} \mathcal{L}_{\text{InfoNCE}}
+
\lambda_{\text{reg}} \mathcal{L}_{\text{NLL}}
\label{eq:total_loss}
\end{equation}
where $\lambda_{\text{align}}$ and $\lambda_{\text{reg}}$ weight the two terms. Both
coefficients were set on a held-out validation split and kept fixed thereafter.

Regularization during optimization draws on the variational information bottleneck
\cite{alemi2016vib}. The full training procedure---encoder pretraining, contrastive alignment,
and fusion---was run sequentially, with each stage warm-starting from the previous checkpoint.

Beyond population-level metrics, Organ-Age provides per-subject outputs: individualized
organ-deviation panels (v4) and gene-level attribution summaries that trace organ-age residuals
back to specific transcripts (v4.5).

% ============================================================
% RESULTS
% ============================================================
\section{Results}

I evaluated the model in three configurations---unimodal, na\"ive multimodal fusion, and
contrastively aligned multimodal fusion---to isolate what each component contributes.
Performance is reported as mean absolute error (MAE) and mean squared error (MSE), with
Pearson correlation ($r$) between predicted and chronological age used as a secondary metric.
All metrics are computed on held-out test splits; training and validation data were used
exclusively for model selection and hyperparameter tuning.

\subsection{Population-level behavior in the normative setting}

I first examined each modality in isolation. The RNA encoder, trained on GTEx expression data
($n = 7{,}378$), achieved an MAE of 9.62 years (MSE $= 156.4$). Prediction
variance increased at older ages, consistent with the documented rise in transcriptomic
heterogeneity among older individuals. The encoder captured molecular aging correlates across
tissues but lacked structural context, limiting its ability to resolve anatomical features of
organ aging.

The chest X-ray encoder ($n = 187{,}825$) achieved an MAE of 10.92 years (MSE $= 186.2$,
$r = 0.76$), learning macrostructural aging cues including skeletal morphology, cardiac
silhouette shape, and soft-tissue composition changes. However, predicted ages clustered within
a compressed range around the population mean (effective prediction range $\sim$25--65 years
versus a true span of 18--90), indicating that the model captured a robust population trend
while attenuating individual-level variation. The MRI encoder ($n = 563$) showed the highest
MAE among unimodal baselines at 14.93 years (MSE $= 312.2$), reflecting the challenge of
generalizing structural brain signals to cross-tissue age estimation without a shared latent
space; as shown below, alignment dramatically rescues MRI performance.

\begin{figure}[t]
\centering
\includegraphics[width=\linewidth]{normative_pred_vs_chrono_all.png}
\caption{
\textbf{Overall prediction behavior across the population (normative setting).}
Predicted biological age versus chronological age for the aggregated cohort.
}
\label{fig:pred_vs_chrono_all_normative}
\end{figure}

\begin{figure}[t]
\centering
\includegraphics[width=\linewidth]{normative_gap_vs_chrono_all.png}
\caption{
\textbf{Age-gap behavior across the population (normative setting).}
Difference between predicted biological age and chronological age as a function of chronological
age.
}
\label{fig:gap_vs_chrono_all_normative}
\end{figure}

As a first multimodal baseline, I built a na\"ive fusion model (v3) that concatenates
modality-specific embeddings without prior alignment. Across the full dataset this achieved
an MAE of 10.88 years (MSE $= 185.4$)---comparable to the X-ray unimodal result and worse
than RNA alone, confirming that na\"ive concatenation does not reliably leverage cross-modal
complementarity without prior alignment.
However, UMAP projections of the concatenated embeddings showed that points still clustered
primarily by modality rather than by age (Section~3.2), reflecting unstable fusion behavior.

\begin{figure}[t]
\centering
\includegraphics[width=\linewidth]{normative_gap_hist_all.png}
\caption{
\textbf{Organ-age gap distribution (normative setting).}
Histogram of $\Delta=\hat{y}-\mathrm{Age}_{\text{true}}$ summarizing the distribution of accelerated
(positive) and decelerated (negative) aging deviations across the cohort.
}
\label{fig:organ_age_gap_hist_normative}
\end{figure}

\subsection{Impact of contrastive alignment on multimodal fusion (v3.5)}

Introducing contrastive alignment before the fusion stage reduced inter-modality variance in the
latent space and yielded consistent improvements across all metrics. The resulting model (v3.5)
achieved an MAE of 9.30 years (MSE $= 138.0$, $r = 0.74$) across the full 195{,}766-sample
dataset. Table~\ref{tab:ablation} summarizes the ablation: aligned fusion reduced MAE by 14.6\%
relative to na\"ive fusion. The MRI modality showed the largest per-modality gain, with MAE
dropping from 14.93 to 6.21 years ($-58\%$) after alignment, compared with 14.7\% for X-ray
and 4.8\% for RNA.

\begin{table}[t]
\centering
\caption{
\textbf{Ablation: unimodal, na\"ive fusion, and aligned fusion.}
All metrics are on held-out test splits. Unimodal results use per-modality evaluation of the
na\"ive v3 model on subjects with access to only that input type. $\Delta$MAE reports the
relative MAE reduction from na\"ive fusion (v3) to aligned fusion (v3.5).
}
\begin{tabular}{lccc}
\toprule
\textbf{Configuration} & \textbf{MSE} & \textbf{MAE (yr)} & \textbf{$r$} \\
\midrule
RNA only       & 156.4 &  9.62 & --- \\
X-ray only     & 186.2 & 10.92 & 0.76 \\
MRI only       & 312.2 & 14.93 & --- \\
\midrule
Na\"ive fusion (v3)   & 185.4 & 10.88 & --- \\
Aligned fusion (v3.5) & 138.0 &  9.30 & 0.74 \\
\bottomrule
\end{tabular}
\label{tab:ablation}
\end{table}

The improvement held across all three modalities when evaluated per-modality within the aligned
framework (Table~\ref{tab:v35_results}), indicating that alignment makes each modality a more
consistent contributor to the fused representation rather than one modality dominating. The MRI
subset achieved the lowest MAE (6.21 years), likely reflecting both the higher intrinsic
signal-to-noise of structural brain imaging and the relative homogeneity of the IXI cohort.

\begin{table}[t]
\centering
\caption{
\textbf{Performance of contrastively aligned multimodal fusion (v3.5), by modality.}
MSE and MAE on held-out test splits. Per-modality results confirm that alignment benefits all
input types.
}
\begin{tabular}{lrcc}
\toprule
\textbf{Modality} & \textbf{N} & \textbf{MSE} & \textbf{MAE (yr)} \\
\midrule
All    & 195,766 & 138.00 & 9.30 \\
RNA    &   7,378 & 139.44 & 9.16 \\
X-Ray  & 187,825 & 138.18 & 9.31 \\
MRI    &     563 &  61.92 & 6.21 \\
\bottomrule
\end{tabular}
\label{tab:v35_results}
\end{table}

To visualize what alignment does to the latent space, I projected unaligned and aligned embeddings
with UMAP \cite{mcinnes2018umap}. Before alignment, the first two UMAP components separated
points by modality, with modality identity explaining the majority of embedding variance. After
alignment, embeddings organized along age-related gradients rather than by modality identity,
confirming that the contrastive objective shifted the dominant axis of variation from domain
identity to aging structure.

The bottom-line output is the organ-age gap:
\begin{equation}
\Delta = \hat{y} - \mathrm{Age}_{\text{true}}
\label{eq:agegap}
\end{equation}
In the aligned model, predicted ages tracked chronological age in a tighter band (residual
$\mathrm{SD} = 11.8$ years, versus 13.6 in the na\"ive fusion model), a reduction driven
primarily by alignment's stabilizing effect on the MRI modality. Age-bin analysis reveals
the expected regression-to-mean pattern: prediction errors are highest for the youngest
subjects (20--40 age range: MSE $= 300$) and lowest for the 60--80 group (MSE $= 85$),
consistent with cohort-trained age predictors attenuating extreme values toward the population
mean. This matters because organ-resolved interpretation depends on stable residual structure;
if residuals drift systematically with age or modality availability, organ-age deltas become
difficult to compare across individuals.

\subsection{Calibrated outputs and uncertainty-aware prediction}

I applied post-hoc calibration to put predicted ages on a scale more directly comparable to
chronological age and to make residuals comparable across organs.
Figure~\ref{fig:calibrated_pred_with_ci} shows the result. After calibration, a systematic regression-to-mean bias is evident: predicted biological ages
are over-estimated for young subjects ($+$27 years in the 20--30 bin, $+$18 years in the 30--40
bin) and under-estimated for older subjects ($-$10 years in the 60--70 bin, $-$20 years in the
70--80 bin). This reflects the well-known tendency of cohort-trained age predictors to compress
predictions toward the population mean. The 95\% confidence intervals are approximately
$\pm$20 years across all age bins, providing context for organ-age deviations and motivating
the organ-resolved z-score framing, which accounts for this age-dependent structure.

\begin{figure}[t]
\centering
\includegraphics[width=\columnwidth]{calibrated_pred_vs_chrono_with_ci.png}
\caption{
\textbf{Calibrated organ-age prediction with uncertainty.}
Calibrated predicted biological age with a 95\% confidence interval compared to chronological age.
}
\label{fig:calibrated_pred_with_ci}
\end{figure}

\subsection{Per-organ calibrated prediction behavior}

To quantify that organ aging is not uniform, I break out calibrated results for four
representative organs. Table~\ref{tab:per_organ} reports per-organ MAE, residual standard
deviation, and mean age-gap ($\bar{\Delta}$).

\begin{table}[t]
\centering
\caption{
\textbf{Per-organ prediction metrics (calibrated, v3.5).}
MAE, residual standard deviation (Res.\ SD), and mean age gap ($\bar{\Delta}$) for four
representative organs. Differences in dispersion and bias across organs confirm that aging
is not uniform.
}
\begin{tabular}{lcccc}
\toprule
\textbf{Organ} & \textbf{N} & \textbf{MAE (yr)} & \textbf{Res.\ SD (yr)} & $\bar{\Delta}$ \textbf{(yr)} \\
\midrule
Brain         &     563 &  5.01 &  6.48 & $\phantom{+}$0.0 \\
Brain cortex  &     270 &  7.95 & 10.14 & $\phantom{+}$0.0 \\
Heart         &     913 &  8.19 & 10.85 & $\phantom{+}$0.0 \\
Lung          & 188,429 & 15.50 & 17.29 & $-$6.93 \\
\bottomrule
\end{tabular}
\label{tab:per_organ}
\end{table}

Brain and brain cortex show the tightest residual distributions (SD $= 6.48$ and 10.14 years,
respectively), consistent with the lower measurement noise and more constrained biological
variability of structural neuroimaging. Lung shows the widest dispersion (SD $= 17.29$ years),
reflecting both greater biological heterogeneity---driven in part by exposure variation such
as smoking---and the noisier transcriptomic signal in this tissue. Notably, lung carries a
substantial negative mean age gap ($\bar{\Delta} = -6.93$ years), indicating that predicted
lung ages systematically under-estimate chronological age in this cohort.

Figure~\ref{fig:calibrated_overlay_summary} confirms these patterns visually. All four organs
exhibit roughly monotonic predicted-versus-chronological-age relationships, but they differ in
slope attenuation (brain predictions span $\sim$69\% of the true age range; lung predictions
are heavily compressed), residual structure, and age-gap density shape. These organ-specific
patterns would be invisible in a pooled analysis and motivate the individualized interpretation
that follows.

\begin{figure}[t]
\centering
\includegraphics[width=\linewidth]{calibrated_overlay_summary_key_organs.png}
\caption{
\textbf{Calibrated organ-age overlays for key organs.}
Compact three-view summary for four representative organs (brain, brain cortex, heart, lung):
\textit{top-left} overlay of predicted biological age vs.\ chronological age, \textit{top-right}
overlay of age-gap trajectories $\Delta = \hat{y} - \mathrm{Age}_{\text{true}}$, and
\textit{bottom} density-normalized age-gap distributions.
}
\label{fig:calibrated_overlay_summary}
\end{figure}

\subsection{Exposure-related interpretation: smoking and alcohol burden as organ-local acceleration signals}

Organ-age deltas should be interpreted as correlational signals, not diagnoses. Even so, the
organ-resolved structure makes it possible to separate localized acceleration from global drift,
which is relevant for exposures that disproportionately affect specific organs.

For lungs, chronic smoking is a canonical example. Smoking alters airway and parenchymal structure
over years, producing radiographic patterns and transcriptomic changes tied to inflammation,
extracellular matrix remodeling, and stress-response programs. In a fused representation, these
molecular and structural cues can reinforce one another. An older-appearing lung profile can thus
emerge even when organism-wide age remains near expected, because the signal is localized rather
than systemic.

For liver, chronic alcohol burden and metabolic strain are analogous. Alcohol exposure and
metabolic dysregulation drive transcriptomic shifts in lipid handling, oxidative stress, and
inflammatory pathways, alongside structural consequences that can manifest as altered tissue
composition and downstream systemic effects. The key point is not causal attribution, but that
multi-signal fusion supports a \emph{localized deviation} framework: the model can register a
coherent liver-specific age gap even when other organs remain nominal.

This interpretation has a practical implication: organ-resolved age gaps provide a readable,
organ-local summary that can be used to prioritize follow-up. The output is not a clinical label.
It is a structured flag: \emph{this organ deviates from age-matched expectation by $\Delta$ years
with confidence interval $[\Delta - z_{\alpha}\sigma, \; \Delta + z_{\alpha}\sigma]$}.

\subsection{Hero subject analysis (Vitalis v4): individualized organ-age profiling}

While cohort-level metrics validate overall model behavior, individualized interpretation is the
intended endpoint of Organ-Age. A single ``hero subject'' analysis illustrates how calibrated
organ-age outputs translate into organ deltas, uncertainty intervals, and standardized z-scores.

Subject GTEX-1117F (chronological age $\approx$ 64.5 years) was selected as a representative
case because tissue was available for multiple organs, enabling a cross-organ comparison within
a single individual. Figure~\ref{fig:vitalis_hero_stack} shows the per-organ deviation profile.
Skeletal muscle exhibited the largest negative deviation ($\Delta = -14.1$ years, $z = -1.09$),
while brain cortex showed a more moderate offset ($\Delta = -5.7$ years, $z = -0.56$). All
sampled organs for this subject fell below chronological age, yielding a mean organ-age gap of
$\bar{\Delta} = -10.5$ years across the seven profiled tissues---consistent with a globally
younger-appearing molecular and structural profile.

\begin{figure}[t]
\centering
\includegraphics[width=\linewidth]{1117F_panel.png}
\caption{
\textbf{Hero subject (Vitalis v4): organ-level deviation profile.}
Per-organ age-gap bars for subject GTEX-1117F with organ-wise z-score annotations.
}
\label{fig:vitalis_hero_stack}
\end{figure}

To preserve individualized context, I also show the corresponding uncertainty and shape-profile
diagnostics for the same subject (Figure~\ref{fig:vitalis_hero_ci_radar}). These complementary
views separate a large but uncertain deviation from a smaller but stable one. For instance, the
skeletal muscle deviation of $-14.1$ years carries a wide 95\% CI (absolute bounds
$\sim$22--65 years), warranting caution, whereas the kidney deviation of $-8.3$ years has a
narrower interval (absolute bounds $\sim$35--72 years), indicating a more stable estimate. The radar panel reveals that outlier
burden for this subject is distributed across most organs rather than concentrated in one or two,
supporting interpretation as a systemic rather than organ-local phenomenon.

\begin{figure}[t]
\centering
\begin{subfigure}[t]{0.49\linewidth}
\centering
\includegraphics[width=\linewidth]{ci_plot.png}
\caption{Calibrated organ ages with 95\% CI.}
\end{subfigure}
\hfill
\begin{subfigure}[t]{0.49\linewidth}
\centering
\includegraphics[width=\linewidth]{radar.png}
\caption{Standardized organ z-score profile.}
\end{subfigure}
\caption{
\textbf{Hero subject (Vitalis v4): complementary uncertainty and z-score diagnostics.}
The CI panel shows uncertainty-aware organ ages around the chronological baseline, while
the radar panel summarizes relative organ burden in standardized units.
}
\label{fig:vitalis_hero_ci_radar}
\end{figure}

\subsection{Molecular interpretability (v4.5): gene-level attribution of organ-age signals}

A model that predicts organ age accurately but offers no molecular insight is of limited
scientific value. The v4 panels show \emph{which} organs deviate; the v4.5 extension asks
\emph{why}, by identifying which genes and latent factors drive the predictions. I report two
complementary analyses.

First, gene-ranking analyses list individual genes whose expression patterns are most strongly
associated with predicted organ age, yielding a direct feature-level summary of candidate
contributors. Second, integrated gradients computed over the latent RNA dimensions show how the
model compresses transcriptomic information internally and which latent factors matter most for
each organ. Between them, these views move from black-box prediction toward concrete molecular
hypotheses.

\begin{figure}[t]
\centering
\includegraphics[width=\linewidth]{top_20_genes_rank_overlay.png}
\caption{
\textbf{Gene-level attribution (v4.5): rank overlay of top-gene importance by organ.}
Importance scores across ranks (top 20 genes per organ) are overlaid for representative organs,
enabling compact cross-organ comparison of attribution concentration and tail decay.
}
\label{fig:top20_genes_panel}
\end{figure}

Figure~\ref{fig:top20_genes_panel} shows that attribution is distributed broadly across the top
20 genes for all profiled organs. The top 5 genes account for $\sim$28\% of total attribution
mass among the top 20 for liver, kidney, brain cortex, and heart alike, indicating that no
single gene dominates and that the model integrates signal from many transcriptomic contributors.
Five genes (gene\_202, gene\_394, gene\_891, gene\_919, gene\_931) appear in the top 20 for all
four organs, suggesting a shared core of aging-correlated transcripts, while organ-specific
genes provide additional tissue-contextualized signal.

These rankings should be read as associations with the model output rather than causal drivers of
aging. Even so, they provide a practical starting point for follow-up: pathway enrichment,
comparison to published aging gene sets, and targeted validation in longitudinal or disease-specific
cohorts.

\begin{figure}[t]
\centering
\includegraphics[width=\linewidth]{ig_three_overlay_rna.png}
\caption{
\textbf{Latent-space feature importance (RNA): cross-organ rank overlay.}
Integrated-gradient-derived latent importance over rank for representative organs, highlighting
how attribution mass concentrates in a subset of dimensions.
}
\label{fig:ig_three_panel_rna}
\end{figure}

Figure~\ref{fig:ig_three_panel_rna} shows that attribution is distributed broadly across latent
RNA dimensions rather than concentrated in a small subset. For liver, the top 3 latent
dimensions capture 3.6\% of total integrated-gradient magnitude; for kidney, 4.2\%; for brain
cortex, 4.1\%. This near-uniform distribution across the 256-dimensional latent space suggests
the model spreads transcriptomic aging information broadly rather than compressing it into a
compact bottleneck. Latent dimension 114 appears in the top 5 for liver, kidney, and brain
cortex, representing one shared contributor, while organ-preferential dimensions include 211
and 204 for liver; 144, 215, and 224 for kidney; and 92, 126, and 132 for brain cortex,
consistent with a decomposition into weakly shared systemic factors and organ-specific programs.

\begin{figure}[t]
\centering
\includegraphics[width=\linewidth]{ig_liver_rna_top20.png}
\caption{
\textbf{Latent RNA features driving liver biological age with cumulative overlay.}
Top latent RNA dimensions (bars) and cumulative attribution fraction (line) for liver age prediction.
}
\label{fig:ig_liver_rna_top20}
\end{figure}

Figure~\ref{fig:ig_liver_rna_top20} zooms in on liver, where the cumulative attribution curve
rises gradually: the top 5 dimensions capture 5.8\% of total integrated-gradient mass, and
the top 10 reach 10.7\%. This diffuse pattern indicates that liver age predictions draw on a
broad transcriptomic signal distributed across the latent space rather than a compact
bottleneck. Each set of top dimensions nonetheless offers an entry point for deeper
investigation through gene loadings, pathway correlations, and comparison to known hepatic
aging programs.

Taken together, v4 and v4.5 connect organ-age deviations to plausible molecular contributors,
moving the framework from prediction toward hypothesis generation.

% ============================================================
% DISCUSSION
% ============================================================
\section{Discussion}

The experiments support a straightforward conclusion: combining molecular and radiological data
in a shared representation improves organ-level biological age estimation. The key ingredient is
contrastive alignment. Without alignment, embeddings occupy distinct regions of latent space, and
the fusion module integrates them inconsistently. With alignment, transcriptomic, chest-X-ray, and
MRI features intermix and organize along age gradients, which makes fusion behavior more stable.
This echoes a broader result in multimodal learning: representation alignment is often a
prerequisite for reliable cross-modal integration \cite{baltrusaitis2018survey}.

Quantitatively, aligned fusion reduced MAE by 14.6\% over na\"ive fusion
(Table~\ref{tab:ablation}). The gain was not driven by a single modality:
per-modality evaluation within the aligned framework showed consistent improvements across RNA
(4.8\%), X-ray (14.7\%), and MRI (58\%) inputs (Table~\ref{tab:v35_results}), indicating that
alignment makes each data type a better contributor to the joint representation. The residual
standard deviation dropped from 13.6 years (na\"ive fusion) to 11.8 years (aligned fusion).
These improvements in residual stability are prerequisite for reliable organ-resolved
interpretation, because organ-age deltas are only meaningful if the residual structure is
consistent across the age range.

Relative to existing biological age predictors, Organ-Age adds organ-level resolution. Epigenetic
and transcriptomic clocks quantify systemic molecular aging but are blind to anatomy; imaging-based
models capture structural aging but have no direct access to underlying regulatory state
\cite{putin2016deepaging}. By jointly modeling both domains, Organ-Age yields organ-age gaps that
reflect structural \emph{and} regulatory change. The per-organ metrics (Table~\ref{tab:per_organ})
confirm that different organs exhibit distinct prediction profiles: brain achieves the tightest
residuals (SD $= 6.48$ years) while lung shows the widest (SD $= 17.29$ years), a 2.7$\times$
difference that would be invisible in an organism-wide clock.

The organ-age deltas behave as structured residuals rather than pure error. Different organs show
different dispersion profiles, and residual variance grows differently with age across tissues.
This supports a practical framing: biological age is not a single scalar property uniformly expressed
across the body. It is a distributed quantity with organ-specific trajectories shaped by exposure,
repair capacity, and tissue-specific programs.

From a translational perspective, organ-level deltas offer a localized deviation signal that can
complement organism-wide clocks. Exposures such as smoking and heavy alcohol use disproportionately
affect lungs and liver. In principle, organ-resolved deltas could flag localized acceleration even
when systemic biological age remains near expected. This is not a diagnosis and cannot be treated as
one. It is a prioritization signal: an organ-specific deviation that warrants follow-up in settings
where metadata and longitudinal outcomes are available.

Several limitations deserve mention. The three datasets do not overlap at the individual level, so
alignment operates on representations rather than matched samples; this increases flexibility but
complicates causal interpretation across modalities. The framework currently covers adult aging only
and does not model development. While the probabilistic head yields uncertainty estimates, separating
epistemic from aleatoric uncertainty remains difficult in practice. The sample size imbalance across
modalities is substantial (187{,}825 X-ray samples versus 563 MRI), which means that X-ray
dominates the training signal in the fused model; the strong MRI-specific MAE (6.21 years) should
be interpreted in light of the smaller and more homogeneous IXI cohort rather than taken as
representative of MRI performance at scale.

Natural next steps include adding proteomics, metabolomics, and longitudinal clinical data as
additional modalities; evaluating the model longitudinally to test whether organ-age deviations
predict outcomes; and expanding interpretability analyses to connect latent factors to pathways and
cell-type composition changes.

% ============================================================
% CONCLUSION
% ============================================================
\section{Conclusion}

I have presented Organ-Age, a framework that estimates organ-level biological age by aligning and
fusing transcriptomic and radiological representations. Contrastive alignment pulls modality-specific
embeddings into a shared latent space; a transformer-based fusion module then combines them to yield
organ-resolved age predictions with uncertainty.

On a dataset of over 190{,}000 samples, the aligned multimodal model outperformed unimodal encoders
and an unaligned fusion baseline in predictive accuracy and latent-space coherence. The resulting
organ-age deltas expose structured acceleration and deceleration patterns that differ by organ and
are biologically plausible, illustrating the value of combining molecular and structural perspectives.

The v4 per-subject panels and v4.5 attribution summaries extend the framework beyond prediction:
panels distill organ-level deviations into a readable report, while attribution traces deviations to
candidate transcriptomic drivers and latent factors. Together, they move Organ-Age from a predictive
tool toward a hypothesis-generating platform for studying how individual organs age.

More broadly, these results argue that biological age is better modeled as a multi-signal,
organ-resolved quantity than as a single number. Organ-Age provides one concrete instantiation of
this idea and a starting point for richer, more interpretable aging models.

\bibliographystyle{unsrt}
\bibliography{references}

\end{document}